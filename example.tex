\documentclass[timestamp]{jazzgrid}

\composer{Composer 1, Composer 2}
%\tempo{42}
\title{Example}

\begin{document}
\maketitle
%% Prints the section name on the left side and draws a horizontal line at the end
%%              Name of section
%%                     |
\begin{musicsection}{Four}
%% Creates a line with four bars
\barline
%%%%%%%% Creates a bar with four beats
%%%%%%%% Parameter for \nvolta, \coda and \codasmall
%%%%%%%%          |
	{\barfour{ }
%%%%%%%%%%%%%%%% Creates the same chord using simplified syntax
%%%%%%%%%%%%%%%% chord root                 tone/chord to be played under, same order
%%%%%%%%%%%%%%%%    | sharp/flat (#/b)                  |
%%%%%%%%%%%%%%%%    +---+|   diminished 7th (maj/dim)   |
%%%%%%%%%%%%%%%%        ||minor/ | major 7th            |
%%%%%%%%%%%%%%%%        ||major  |                      |
%%%%%%%%%%%%%%%%        ||| +----+   additional notes   |
%%%%%%%%%%%%%%%%        ||| |  +---------+              |
%%%%%%%%%%%%%%%%        ||| |  |  +---------------------+
		{\chord{Bbmmaj 5- 2+/D}}
		{}
		{\chord{A#}}
		{\chord{Ab}}
	}
	{\barfour{}
		{\chord{Bbmmaj 5- 2+/D}}
		{}
		{\chord{B#}}
		{\chord{Bb}}
	}
	{\barfour{}
		{\chord{C}}
		{}
		{\chord{C#}}
		{\chord{Cb}}
	}
	{\barfour{}
		{\chord{Dbm 9 7 5-}}
		{\chord{D#m 11 7}}
		{\chord{Dbm 9 7 5-}}
		{\chord{D#m 13 9}}
	}
\barline
%%%%%%%%%%%%%%%%%% Creates a coda symbol (in the first quarter)
%%%%%%%%%%%%%%%%%%     position in bar
%%%%%%%%%%%%%%%%%%           |
%%%%%%%%%%%%%%%%%%           | number of positions in bar
%%%%%%%%%%%%%%%%%%           |  |
	{\barfour{\codasmall{1}{4}}
		{\chord{E}}
		{}
		{\chord{E#}}
		{\chord{Eb}}
	}
	{\barfour{}
		{\chord{F}}
		{}
		{\chord{F#}}
		{\chord{Fb}}
	}
	{\barfour{}
		{\chord{G}}
		{}
		{\chord{G#}}
		{\chord{Gb}}
	}
	{\barfour{}
		{\chord{Abm 9 7 5-}}
		{}
		{\chord{B#mmaj 5+}}
		{\chord{Cbdim}}
	}
\end{musicsection}

\begin{musicsection}{Three}
\barline
%%%%%%%% Creates bar with three beats
%%%%%%%% Parameter for \nvolta, \coda and \codasmall
%%%%%%%%           |
	{\barthree{ }
		{\chord{A 1 2}}
		{\chord{B 2 2}}
		{\chord{C 3 2}}
	}
	{\barthree{}
		{\chord{D 4 5}}
		{\chord{E 5 5}}
		{\chord{F 6 5}}
	}
	{\barthree{}
		{\chord{G 7 5}}
		{}
		{}
	}
	{\barthree{}
		{\chord{Cm/G}}
		{}
		{}
	}
\barline
	{\barthree{}
		{\chord{A#m 9 1/Gb}}
		{\chord{Bm 7 2/F#}}
		{\chord{Cbm 6 3/Eb}}
	}
	{\barthree{}
		{\chord{Dm 11 5 4/D#}}
		{\chord{E#m 13 4 5/C#}}
		{\chord{Fm 15 3 6/Bb}}
	}
%%%%%%%%%%%%%%%%%%% Creates a nvolta (primavolta in this case)
%%%%%%%%%%%%%%%%%%%    description
%%%%%%%%%%%%%%%%%%%        |
	{\barthree{\nvolta{1.}}
		{\chord{Gdim/A#m}}
		{}
		{}
	}
	{\barthree{}
		{\chord{G}}
		{\chord{F/G}}
		{}
	}
%% Creates a bar line with two bars
%%    alignment - l/left, r/right (default)
%%          |
%%          |
\barlinetwo{right}
	{\barthree{\nvolta{2.}}
		{\chord{G 7 5-/Bb}}
		{}
		{}
	}
	{\barthree{}
		{\chord{Gm 9 7}}
		{}
		{}
	}
\end{musicsection}

%% Section can be slim, which reduces the bar height
\begin{musicsection}[slim]{coda}
\barline
	{\barfour{}
		{\chord{Dm 7}}
		{}
		{}
		{}
	}
	{\barfour{}
		{\chord{G 7}}
		{}
		{}
		{}
	}
	{\barfour{}
		{\chord{Dm 7}}
		{}
		{}
		{}
	}
	{\barfour{}
		{\chord{G 7}}
		{}
		{}
		{}
	}
\end{musicsection}

\begin{musicsection}{Crazy}
\barline
	{\barfour{}
		{\chord{Amaj}}
		{}
		{}
		{}
	}
	{\barfour{}
		{\chord{Adim}}
		{}
		{}
		{}
	}
	{\barfour{}
		{\chord{Amaj}}
		{}
		{}
		{}
	}
	{\barfour{}
		{\chord{Amaj}}
		{}
		{}
		{}
	}
%% Creates a bar line with one bar
%%    alignment - l/left, r/right (default)
%%          |
%%          |
\barlineone{left}
	{\barfour{}
		{\chord{B#maj}}
		{}
		{}
		{}
	}
\barlinetwo{}
	{\barfour{}
		{\chord{Bbmaj}}
		{\chord{Bbmdim}}
		{\chord{Bmaj}}
		{\chord{Bmaj 7 5}}
	}
	{\barfour{}
		{\chord{B#mmaj 7 5/Gbm 7}}
		{}
		{\chord{B#mmaj 7 5/Gbm 7}}
		{\chord{Bbmdim 7 5/G#m 7}}
	}
%% Creates a bar line with three bars
%%    alignment - l/left, r/right (default)
%%              |
%%              |
\barlinethree{left}
	{\barfour{}
%%%%%%%%%%%%%%%% Creates a rest
%%%%%%%%%%%%%%%%    rest duration
%%%%%%%%%%%%%%%%       |
%%%%%%%%%%%%%%%%       |
		{\rest{1}}
		{\rest{2}}
		{\rest{4}}
		{\rest{8}}
	}
	{\barfour{}
		{\rest{16}}
		{}
		{\rest{whole}}
		{\rest{half}}
	}
	{\barfour{}
		{\rest{quarter}}
		{\rest{eighth}}
		{\rest{sixteenth}}
		{}
	}
\end{musicsection}


\begin{musicsection}{8-bar}

%% Creates a bar line with eight bars
%% \bareight* should be used for bars, though three and four beat bars may not fit inside
\barlineeight
	{\bareighttwo{}{\chord{Gm}}{}}
	{\bareighttwo{}{}{}}
	{\bareighttwo{}{\chord{F#}}{}}
	{\bareighttwo{}{\chord{Bb}}{}}
	{\bareighttwo{}{\chord{Eb}}{\chord{Eb/D}}}
	{\bareighttwo{}{\chord{Eb/C}}{\chord{Eb/Bb}}}
	{\bareighttwo{}{\chord{F}}{}}
	{\bareighttwo{}{}{}}
\barlineeight
	{\bareighttwo{}{\chord{Gm}}{}}
	{\bareighttwo{}{\chord{Gm/F#}}{}}
	{\bareighttwo{}{\chord{Gm 7/F}}{}}
	{\bareighttwo{}{\chord{C sus2}}{}}
	{\bareighttwo{}{\chord{Eb}}{}}
	{\bareighttwo{}{\chord{Edim}}{}}
	{\bareighttwo{}{\chord{Bb}}{}}
	{\bareighttwo{}{\chord{Dm}}{\chord{Bb 5+}}}
\barlineeight
	{\bareighttwo{}{\chord{Gm}}{}}
	{\bareighttwo{}{}{}}
	{\bareighttwo{}{}{}}
	{\bareighttwo{}{}{}}
	{\bareighttwo{}{}{}}
	{\bareighttwo{}{}{}}
	{\bareighttwo{}{}{}}
	{\bareighttwo{}{}{}}
\end{musicsection}

\widthof{Ahoj}

\end{document}
